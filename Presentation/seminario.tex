\documentclass[handout, 11pt]{beamer}

\usepackage[english] {babel}                 % Idioma (és important a l'hora de separar una paraula a final de línia)
%\input{myStyle.sty}


\usetheme{CambridgeUS}
\usepackage{color}
\usepackage{enumitem}
\usepackage{amssymb,mathtools, amsmath, amsfonts, amsthm}
\newcommand{\Ln}{\lim\limits_{n\to \infty}}
\setbeamertemplate{itemize items}[circle]

\title[First Order Logic of Sparse Random Graphs]{First Order
Logic of Sparse Random Graphs}


\begin{document}
	\frame{\titlepage}

\begin{frame}{Preliminaries: The first order logic (FO) of graphs}
	Variables $x_1, \dots, x_n, \dots$ $\rightarrow$ Vertices\\
	A binary relation symbol $E$ $\rightarrow$ Edges\\
	Boolean connectives $\wedge, \vee, \neg, \dots$ and equality symbol $=$.\\
	Quantifiers $\forall, \exists$.\\~\\
	
	Can express
	the existence of a given subgraph,
	the existence of a covering set of size $k$\dots
	\\~\\	
	Cannot express connectivity, $k$-colorability, existence of a Hamiltonian
	path\dots
	\\~\\
	Example: 
	$\exists x \exists y (Exy \wedge \forall z(Ezy \iff Ezx))$.
\end{frame}

\begin{frame}{Preliminaries: The binomial model of random graphs $G(n,p)$}

	Start with vertex set $[n]=\{1,\dots, n\}$ and
	each edge is added with probability $p$ independently. 
	Given $G=([n],E)$,
	\[ \mathrm{Pr}(G)= p^{|E|}\cdot (1-p)^{\binom{n}{2}- |E|}. \]
~\\
We are interested in the asymptotic behavior of 
$G(n,p)$ when $p=p(n)=c/n$.
\end{frame}

\begin{frame}{Preliminaries: The binomial model of random graphs $G(n,p)$}


\begin{theorem}[Erd\"os, R\'enyi, 1960]
	For $c<1$, a.a.s the connected components in
	$G_n$ have size $O(\log(n))$ and are either
	trees or unicycles.\par
	For $c > 1$, a.a.s there is a unique large 
	complex component in $G_n$ of size $O(n)$.	 
\end{theorem}

\begin{theorem}[Lynch, 1960] 
	Let $P$ be a F.O. property of graphs. Then,
	the function
	\[
	F_{P}(c)= \Ln \mathrm{Pr}\big( P(G_n(c)) \big)
	\]
	is well defined and analytic.
\end{theorem}
\end{frame}


\begin{frame}{The problem}
	F.O. properties cannot individually
	detect the phase transition
	at $c=1$.\\~\\
	What if we consider the set $\overline{L(c)}$? Where
	\[L(c)=\{ \Ln \mathrm{Pr}\big( 
	P(G_n(c))\big) \, | \, P \text{ is a F.O. property }.	\}
	\]
	Previous related work:
	\textit{Logical limit laws for minor-closed classes of graphs}
	by P. Heinig, T. Muller, M. Noy, A. Taraz. 
	
\end{frame}

\begin{frame}{Supercritical phase}
$X_{\leq k}(G):= \#$ of cycles of length at most $k$ in $G$.
\begin{theorem}
	\[
	X_{\leq k}(G_n)\xrightarrow[n\to \infty]{D}
	\mathrm{Pois}(\lambda_k),
	\]
	where
	\[
	\lambda_k=\sum_{i=3}^{k} \frac{c^i}{2i}.
	\]
\end{theorem}
~\\
For $c\geq 1$, $\lambda_k \xrightarrow[k\to \infty]{} \infty$.
\\~\\
The property $X_{\leq k}(G)=a$ can be written in F.O. logic. 
\end{frame}  

\begin{frame}{Supercritical phase}
	Because of the Central Limit Theorem one can show that
	\[
	\lim\limits_{x\to \infty}
	\mathrm{Pr}\big( \mathrm{Pois}(x)\leq x + y\sqrt{x}\big) 
	= \Phi(y),	
	\]
	where $\Phi(y)$ is the CDF of $\mathrm{N}(0,1)$.
	\\~\\
	\begin{theorem}
		For $c\geq 1$, $\overline{L(c)}=[0,1]$.
	\end{theorem}
\end{frame}

\begin{frame}{Subcritical phase.}
 $F(G):=$ Union of unicyclic components of $G$.\\~\\
 $\mathcal{U}:=$ Class of graphs whose components are unicycles.\\~\\
Let $P$ be a F.O. property. One
can show via combinatorial games that for any $H\in \mathcal{U}$
\[
\Ln \mathrm{Pr}\big( P(G_n) \,|\, F(G_N)\simeq H_i \big)
= 0 \text{ or } 1.
\]
% Consider $H_1,\dots,H_i,\dots$ an enumeration of $\mathcal{U}$.
% \\~\\
% Define $p_i:=\Ln \mathrm{Pr}\Big( F(G_n)\simeq H_i\Big)$, and
% $S(c):=\{ \sum_{i\in T} p_i(c) \, | \, T\subset \mathbb{N}\, \}$.
% \\~\\
% 
\end{frame}

\begin{frame}{At least one gap for $c<c_0$}
	$A(c):=\Ln \mathrm{Pr}\big( G_n \text{ is acyclic }\big)$\\~\\
	\begin{theorem}[Erd\"os, R\'enyi, 1960]
		\[ A(c)=e^{\frac{2c+c^2}{4}}\sqrt{1-c}. \]	
	\end{theorem}
	~ \vfill
	Let $c_0$ be the only solution to $A(c_0)=1/2$ in $[0,1]$.\\~\\
	\begin{theorem}
		If $c<c_0$ then $1/2\notin \overline{L(c)}$.
	\end{theorem}
	
	
\end{frame}

\begin{frame}{The set of limit probabilities of the fragment}
	 Consider $H_1,\dots,H_i,\dots$ an enumeration of $\mathcal{U}$.
	 \\~\\
	 Define $p_i:=\Ln \mathrm{Pr}\Big( F(G_n)\simeq H_i\Big)$, and
	 $S(c):=\{ \sum_{i\in T} p_i(c) \, | \, T\subset \mathbb{N}\, \}$.
	 \\~\\
	 \begin{theorem}
	 $\overline{L(c)}=S(c)$.
	 \end{theorem}
	 
\end{frame}


\begin{frame}{}
We show an sketch of the proof. 
\vspace*{2mm}
\begin{itemize}[leftmargin=5mm]
	\item[\textbullet] $\overline{L(c)}\subseteq S(c)$
\end{itemize}
\vfill
Let $P$ be a F.O. property. Recall that for any $H_i$
\[
\Ln \mathrm{Pr}\big( P(G_n) \,|\, F(G_N)\simeq H_i \big)
= 0 \text{ or } 1.
\]
In consequence, if $T$ is the set of $i$'s that make last limit
$1$, 
\[
\Ln \mathrm{Pr}\big( P(G_n)\big): = \sum_{i\in T} p_i
\]
\end{frame}
\begin{frame}{}
\begin{itemize}[leftmargin=5mm]
	\item[\textbullet] $\overline{L(c)}\supseteq S(c)$
\end{itemize}
\vfill
The property $F(G)\simeq H_i$ cannot be expressed in F.O. \\~
\\
$F_k(G):=$ Union of the unicyclic components of $G$ whose size is at
most $k$. \\~\\
The property $F_k(G)\simeq H_i$ can be expressed in F.O., and
\[
\lim\limits_{k\to \infty} \Ln
\mathrm{Pr}\big( F_k(G_n)\sim H_i \big) = p_i 
\]
\end{frame}

\begin{frame}{Kakeya criterion for partial sums}
  Fix $0\leq c < 1$. Suppose that
  $p_1\leq p_2\leq \dots\leq p_i\leq\dots$
   \vfill
  \begin{theorem}[J. E. Nymann, R.A S\'aenz, 2000]
  ~	\vfill
  \begin{itemize}[leftmargin=5mm]
  	\item If $p_i\leq \sum_{j=i+1}^\infty$ for all $i$ then
  	$S(c)=[0,1]$.\\~\\
  	
  	\item If $p_i\leq \sum_{j=i+1}^\infty$ for all sufficiently 
  	large $i$ then $S(c)$ is a finite disjoint union of closed
  	intervals.\\~\\
  	
  	\item If $p_i > \sum_{j=i+1}^\infty$ for all sufficiently large
  	$i$	then $S(c)$ is homeomorphic to the Cantor set.
	\end{itemize}
	\end{theorem}
\end{frame}

\begin{frame}{Probability distribution of the fragment}
	\begin{theorem}
		Let $H\in \mathcal{U}$. Then
		\[
		\Ln \mathrm{Pr}\big( F(G_n)\simeq H  \big)
		= A(c) \cdot \frac{s(c)^{v(H)}}{|Aut(H)|},
		\]
		where $s(c)=ce^{-c}$, and 
		\[
		A(c)=\Ln \mathrm{Pr}\big( G_n \text{ is acyclic }\big)
		= e^{\frac{2c+c^2}{4}}\sqrt{1-c}.
		\] 		
	\end{theorem}
\end{frame}

\begin{frame}
	\begin{theorem}
		$\overline{L(c)}$ is always a finite union of intervals
	\end{theorem}
\vfill
	If $A\cdot s^{k-1}\geq p_i> A\cdot s^k$ we can prove that
	\[
	\sum_{j=i+1}^\infty p_j\leq A\cdot s^{k+1}\frac{k-3}{2}.\]
	Thus, if $\frac{k-3}{2}\geq \frac{1}{s}$, then
	\[
	p_i \geq \sum_{j=i+1}^\infty p_j.
	\]
\end{frame}

\begin{frame}
\begin{theorem}
	For $c\geq c_0$\\
	$\overline{L(c)}=[0,1]$
\end{theorem}
\vfill
Using our previous bound and $s(c)>1/3$ we get that
if $A\cdot s^{k-1}\geq p_i> A\cdot s^k$ for  $k\geq 9$, then
\[
p_i \geq \sum_{j=i+1}^\infty p_j.
\]
\\~\\
After this we have to check only finitely many $p_i$'s.
\end{frame}



\begin{frame}{More results}
	One can show that the number of gaps in $\overline{L(c)}$
	tends to infinity as $c$ goes to zero.
	\vfill
	These results (with the appropriate changes) also hold in the setting of random uniform
	hypergraphs.  
\end{frame}

\begin{frame}{Recap}
	For $c\geq 1$ all probabilities can be approximated with statements of the type ``$G_n$ contains less than $k$ cycles of order less than $l$".\\~\\
	For $c<1$ we can approximate F.O. probabilities using sums of fragment
	probabilities. \\~\\
	Using the Kakeya criterion one can show that there is a gap in
	$\overline{L(c)}$ only when the asymptotic probability of $G_n$ being acyclic is 
	greater than $1/2$.
	
\end{frame}
\end{document}