\documentclass[11pt,notitlepage,a4paper]{article}

\usepackage[left=2cm,right=2cm,top=2cm,bottom=2cm]{geometry}
\usepackage{graphicx}
%%BeginIpePreamble
\usepackage{amssymb,mathtools, amsmath, amsfonts, amsthm}
%%EndIpePreamble
%\usepackage{color}
\usepackage{float}
\usepackage{hyperref}
\usepackage{enumerate}
\usepackage{enumitem}
\usepackage{chngcntr}
\usepackage{cleveref}
\usepackage{pdfpages}
\usepackage{caption,subcaption,float}
\usepackage[utf8]{inputenc}

\counterwithout{equation}{section}



\newlength{\margen}
\setlength{\margen}{\paperwidth}
\addtolength{\margen}{-\textwidth}
\addtolength{\skip\footins}{0.7 cm}
\setlength{\margen}{0.5\margen}
\addtolength{\margen}{-1in}
\setlength{\oddsidemargin}{\margen}
\setlength{\evensidemargin}{\margen}
\setlength{\abovedisplayskip}{3pt}
\setlength{\belowdisplayskip}{3pt}
%%%% Small setup %%%%
\hypersetup{
	colorlinks=false,
	pdfborder={1 1 0.0005},
}
\setlength{\parskip}{0.2cm}
%%%%%%%%%%%%%%%
%\usepackage{tikz-cd}
%\usetikzlibrary{cd}
\usepackage[english]{babel}
\usepackage{todonotes}
%\usepackage{cleveref}
%\usepackage{caption}
%\usepackage{subcaption}
%\usepackage{bbding}
%\usepackage{tcolorbox}
%\usepackage{natbib}
%
%\DeclareMathOperator{\incl}{incl}

\newtheorem{proposition}{Proposition}[section]
\newtheorem{fact}{Fact}[section]
\newtheorem{theorem}{Theorem}[section]
\newtheorem{lemma}{Lemma}[section]
\newtheorem{corollary}{Corollary}[section]
\theoremstyle{definition}
\newtheorem{definition}{Definition}[section]
\newtheorem{propdef}{Proposition / Definition}[section]
\newtheorem{remark}{Remark}[section]
\newtheorem*{lemma*}{Lemma}
\newtheorem*{claim*}{Claim}

%
%\newtheorem{inneraxiom}{Axiom}
%\newenvironment{axiom}[1]
%{\renewcommand\theinneraxiom{#1}\inneraxiom}
%{\endinneraxiom}
%\newcommand{\cc}{\mathfrak{c}}


%\newcommand{\Hc}{\mathcal{H}}
%\newcommand{\Lan}{\mathcal{L}}
%
%\newcommand{\clist}{\mathfrak{c}_{1}, \cdots, \mathfrak{c}_m}
%\newcommand{\morph}[1]{\stackrel{#1}{\simeq}}
%\newcommand{\vlst}[2]{#1_1,\dots, #1_{#2}}
%\newcommand{\gnp}{G(n,\beta_1/n^{a_1-1}, \dots,\beta_l/n^{a_l-1})}
\newcommand{\Z}{\mathbb{Z}}
\newcommand{\CC}{\mathbb{C}}
\newcommand{\Q}{\mathbb{Q}}
\newcommand{\R}{\mathbb{R}}
\newcommand{\N}{\mathbb{N}}
\DeclarePairedDelimiter\floor{\lfloor}{\rfloor}
\newcommand{\Ln}{\lim\limits_{n\to \infty}}
\newcommand{\LN}{\lim\limits_{N\to \infty}}

\title{More proof Sketches Regarding the Closure of 
	Limiting Probabilities in Sparse Random Hyper-graphs}
\date{\today}
\author{Alberto Larrauri}


\begin{document}
	\maketitle 
\section{Preliminaries}
$G^d(N,p)$ denotes the binomial model of random $d$-uniform hyper-graphs. 
Consider $d\geq 2$ fixed for the rest of this writing. 

\begin{theorem} Let $c<(d-2)!$. Then, a.a.s all the components
of $G^d(N,p(N))$, where $p(N)\sim c/N^{d-1}$, are either trees or unicycles.
\end{theorem}
\begin{proof} See \cite{erdHos1960evolution} for $d=2$ and
	\cite{karonski2002phase} for the general case.
	\todo[inline]{Está todo hecho en el modelo uniforme, 
		pero la transferencia uniforme-$>$ binomial es ``sencilla''}
\end{proof}

\begin{theorem}
	Let $c<(d-2)!$ and let $p(N)\sim c/N^{d-1}$. Let $\delta_N$ be the
	random variable that counts the vertices in $G^d(N,p(N))$ belonging
	to some unicyclic component. Then there is some constant $C$ such that 
	$\mathrm{E}[\delta_n]$ is smaller than $C$ for any $N$. 
	\todo[inline]{Este teorema lo he visto para grafos en el artículo de Erdos de 1960
	, y más completo en el libro de Karonski, por ejemplo. Para hypergrafos no lo he encontrado, pero igual es cosa de buscar mejor. Lo demuestro de todas formas.}
\end{theorem}
\begin{proof}
	Let $C_d(m)$ denote the number of connected labeled $d$-uniform
	hypergraphs with $m$ edges and $n:=(d-1)m$ vertices. It is satisfied 
	that for big values of $m$:
	\[C_d(m)\leq \frac{n^n}{e^{(m-n)}}\frac{1}{(d-2)!^m},\]
	where the constant hidden by the $O$-notation depends only on $d$.
	Let $X_{N,m}$ be the random variable that counts the number of
	unicyclic components in $G^d(N,c/N^{d-1})$ with exactly $m$ edges.
	Then we have $\delta_N =\sum_{m=2}^{N/(d-1)} (d-1)m \cdot X_{N,m}$.
	Also,
	\begin{align}
	\nonumber \mathrm{E}[X_{N,m}] & \leq 
	\binom{N}{n} p(N)^m
	(1-p(N))^{\binom{N}{d}-\binom{N-n}{d} - m}  \frac{n^n}{e^{(n-m)}}\frac{1}{(d-2)!^m}\\ 
%	& \leq \frac{N^n}{n!}e^{\frac{n(n-1)}{2N}}p(N)^m
%	e^{-p(N)\left[\binom{N}{d}-\binom{N-n}{d} - m\right]}
%	\frac{n^n}{e^{(m-n)}}\frac{1}{(d-2)!^m}\\
	&\nonumber \leq \frac{N^ne^n}{n^n}e^{-\frac{n(n-1)}{2N}}p(N)^m
	e^{-p(N)\left[\binom{N}{d}-\binom{N-n}{d} - m\right]}
	\frac{n^n}{e^{(n-m)}}\frac{1}{(d-2)!^m}\\
	&\label{eqn:bound} =\left(\frac{c}{(d-2)!}\right)^m e^{n-\frac{n(n-1)}{2N}
	-p(N)\left[\binom{N}{d}-\binom{N-n}{d} - m\right]+ 
	(m-n)} 
	\end{align}
	Operating on the exponent of $e$ in the last term:
	\begin{align*}
	&n-\frac{n(n-1)}{2N}
	-p(N)\left[\binom{N}{d}-\binom{N-n}{d} - m\right]+ 
	(m-n)\leq\\ 
	&- \frac{n(n-1)}{2N} - p(N)\left[\binom{N}{d}-\binom{N-n}{d}\right]
	+ p(N)m + m \leq \\
	& m\Big(-\frac{(d-1)}{2} - \frac{c}{(d-2)!} + p(N) + 1\Big)	\leq \\
	& m\Big(1 - \frac{c}{(d-2)!} \Big).	
	\end{align*}
	Thus, substituting in \cref{eqn:bound} we obtain
	\begin{equation}
	\label{eqn:expbound}
		\mathrm{E}[X_{N,m}]\leq \Bigg(\frac{c}{(d-2)!}
		e^{1-\frac{c}{(d-2)!}} \Bigg)^m
	\end{equation}
	One can easily check that $xe^{1-x}$ grows monotonously from $0$
	to $1$ as $x$ goes from $0$ to $1$. Thus, $\frac{c}{(d-2)!}
	e^{1-\frac{c}{(d-2)!}}<1$.\par
	By definition it is satisfied
	\[
	\mathrm{E}[\delta_N]=\sum_{m=2}^{N/(d-1)}	m \mathrm{E}[X_{N,m}],
	\]
	except for the case $d=2$, where the sum starts at $m=3$ instead of $m=2$.
	This is because there are no connected graphs with $2$ vertices and $2$ edges. 
	Because of \cref{eqn:expbound}, for sufficiently large values of $m$ the terms
	$m\mathrm{E}[X_{N,m}]$ are bounded by
	$m(\frac{c}{(d-2)!}e^{1-\frac{c}{(d-2)!}})^m$ uniformly for all values of $N$. 
	Otherwise, for small values of $m$, the $m\mathrm{E}[X_{N,m}]$ are bounded as
	well because
	\[
	\LN m\mathrm{E}[X_{N,m}]= \left(ce^{\frac{c}{(d-2)!}}\right)^m\frac{C_d(m)}{(m\cdot(d-1))!}
	\]
	In consequence, using the dominated converge theorem we can conclude that
	\[
	\LN \mathrm{E}[\delta_N]=\sum_{m=2}^{N/(d-1)}	m \mathrm{E}[X_{N,m}]
	\]
	exists and is not infinite. 
	\end{proof}
	
	\begin{theorem}
		Let $c< (d-2)!$, and let $p(N)\sim c/N^{d-1}$. For any $k\geq 2$, let  $\gamma_{N,\geq k}$ be the random variable that counts how many 
		cycles with at least $k$ edges lie in $G^d(N,p(N))$. Then
		\[
		\LN \mathrm{E}[\gamma_{N,\geq k}]=\sum_{l=k}^{\infty} 
		\left(\frac{c}{(d-2)!}\right)^l \frac{1}{2l}
		\] 	
		In particular, if $\gamma_N$ is the random variable that counts the cycles
		in $G^d(N,p(N))$ then, if $d>2$:
		\[
		\LN \mathrm{E}[\gamma_{N}]	= 
		\begin{cases}
		\frac{c}{2(d-2)!}+ \ln\left(1-\frac{c}{2(d-2)!}\right), &  
		\text{ if } d>2\\
		\frac{c}{2}+ \frac{c^2}{4} 
		\ln\left(1-\frac{c}{2}\right), &
		\text{ if } d=2.
		\end{cases}
		\]
	\end{theorem}
	\todo[inline]{Esto está hecho en el artículo de Erdos para grafos, y para 
	hypergrafos no lo he visto.}
	\begin{proof}
		For $k\geq 2$ ($k\geq 3$ if $d=2$), let $\gamma_{N,k}$ be the random
		variable that counts
		the $k$-cycles that lie in $G^d(N,p(N))$.
		A simple computation yields
		$
		\mathrm{E}[\gamma_{N,k}]= \frac{(N)_{k(d-1)}}{2k}
		\left(\frac{c}{N^{d-1}}\right)^k,$
		and
		$
		\LN	\mathrm{E}[\gamma_{N,k}]= \frac{c^k}{2k}
		$.
		Not only this, but also
		$
		\mathrm{E}[\gamma_{N,k}]\leq \frac{c^k}{2k}$
		for all $N$. In consequence, applying the dominated convergence
		theorem we obtain
		\[ 
		\LN \sum_{l=k}^\infty \mathrm{E}[\gamma_{N,l}]=
		 \sum_{l=k}^\infty \frac{c^l}{2l}.
		\]
		Using that $\gamma_{N,\geq k}$ is the sum of all the $\gamma_{N,l}$
		for $l\geq k$ and the Taylor expansion of $ln(1-x)$ yields the desired
		results. 
	\end{proof}
	
	Given $0<c<(d-2)!$ we define the function $B_d(c)$ as 
	the limit as $N$ tends to infinity of the expected number of
	cycles in $G^d(N,c/N^{d-1})$. Because of our last theorem
	\[
	B_d(c)=\begin{cases}
	\frac{c}{2(d-2)!}+ \ln\left(1-\frac{c}{2(d-2)!}\right)
	,& \text{ if } d>2\\	
	\frac{c}{2}+ \frac{c^2}{4} \ln\left(1-\frac{c}{2}\right)
	, &\text{ if } d=2.
	\end{cases}
	\]
	
	
	A simple application of the factorial moments method proves the following
	theorem:
	\begin{theorem}
		Let $k_1,\dots,k_j\geq 2$ ($\geq 3$ if $d=2$). Then, 
		as $N$ tends to infinity the $\gamma_{N,l_i}$'s converge 
		in distribution to independent Poisson variables
		with mean values $\lambda_i:=\frac{c^{l_i}}{2l_i}$ respectively.
		That is, for any $a_1,\dots,a_j\in \N$,
		\[
		\LN \mathrm{Pr}\Big(
		\bigwedge_{i=1}^r
		\gamma_{N,l_i}=a_i		
		\Big)= \prod_{i=1}^{j} e^{-\lambda_i}\frac{\lambda_i^{a_i}}{a_i!}
		\]
	\end{theorem}
  	\todo[inline]{Esto está hecho en el libro de Bollobás de random graphs. 
  	Para hypergrafos
  	no lo he encontrado, pero realmente es lo mismo. Si queréis lo hago.}
	
	
	Let $H$ be an unicycle and let $m=|E(H)|$, $n=|V(H)|=m(d-1)$. 
	Let $X_{N,H}$ be the random variable that counts
	the number of connected components in $G^d(N,p(N))$ isomorphic to $H$. Then
	\[ 
	\mathrm{E}[X_{N,H}]= \frac{(N)_n}{|Aut(H)|}
	p(N)^{m}(1-p(N))^{\binom{N}{d}-\binom{N-n}{d} - m}.
	\]
	And if $p(N)\sim c/N$,
	\[
	\LN \mathrm{E}[X_{N,H}]=  \frac{c^m}{|Aut(H)|}
	e^{\LN c\frac{\binom{N}{d}-\binom{N-n}{d} - m}{N^{d-1}}}
	=  \frac{c^m}{|Aut(H)|}
	e^{c\frac{m}{(d-2)!}}.
	\]	
	For convenience's sake we will often use the auxiliary variable
	$s=\frac{c}{(d-2)!}e^{\frac{c}{(d-2)!}}$. We can rewrite last 
	limit in terms of $s$ as:
	\[
	\LN \mathrm{E}[X_{N,H}]=(s)^m
	\frac{(d-2)!^m}{|Aut(H)|}.	
	\]
	Another application of the factorial moments method proves the next theorem:

\begin{theorem}
	Let $H_1,\dots, H_j$ be unicycles, and let $P(N)\sim c/N$. 
	Then, as $N$ tends to infinity, the $X_{N,H_i}$'s converge in 
	distribution to independent Poisson variables with means
	$\lambda_i=s^{|E(H_i)|}\frac{(d-2)!^{|E(H_i)|}}{|Aut(H_i)}$
	respectively. That is, for any $a_1,\dots, a_j\in \N$,
	\[
	\LN \mathrm{Pr}\Big(
	\bigwedge_{i=1}^r
	X_{N,H_i}=a_i		
	\Big)= \prod_{i=1}^{j} e^{-\lambda_i}\frac{\lambda_i^{a_i}}{a_i!}
	\]
\end{theorem} 

Next we show some stuff




\begin{theorem}
	
\end{theorem} 




 




\bibliography{biblio}
\bibliographystyle{unsrt}
\end{document}