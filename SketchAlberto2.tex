\documentclass[11pt,notitlepage,a4paper]{article}

\usepackage[left=2cm,right=2cm,top=2cm,bottom=2cm]{geometry}
\usepackage{graphicx}
%%BeginIpePreamble
\usepackage{amssymb,mathtools, amsmath, amsfonts, amsthm}
%%EndIpePreamble
%\usepackage{color}
\usepackage{float}
\usepackage{hyperref}
\usepackage{enumerate}
\usepackage{enumitem}
\usepackage{chngcntr}
\usepackage{cleveref}
\usepackage{pdfpages}
\usepackage{caption,subcaption,float}
\usepackage[utf8]{inputenc}

\counterwithout{equation}{section}



\newlength{\margen}
\setlength{\margen}{\paperwidth}
\addtolength{\margen}{-\textwidth}
\addtolength{\skip\footins}{0.7 cm}
\setlength{\margen}{0.5\margen}
\addtolength{\margen}{-1in}
\setlength{\oddsidemargin}{\margen}
\setlength{\evensidemargin}{\margen}
\setlength{\abovedisplayskip}{3pt}
\setlength{\belowdisplayskip}{3pt}
%%%% Small setup %%%%
\hypersetup{
	colorlinks=false,
	pdfborder={1 1 0.0005},
}
\setlength{\parskip}{0.2cm}
%%%%%%%%%%%%%%%
%\usepackage{tikz-cd}
%\usetikzlibrary{cd}
\usepackage[english]{babel}
\usepackage{todonotes}
%\usepackage{cleveref}
%\usepackage{caption}
%\usepackage{subcaption}
%\usepackage{bbding}
%\usepackage{tcolorbox}
%\usepackage{natbib}
%
%\DeclareMathOperator{\incl}{incl}

\newtheorem{proposition}{Proposition}[section]
\newtheorem{fact}{Fact}[section]
\newtheorem{theorem}{Theorem}[section]
\newtheorem{lemma}{Lemma}[section]
\newtheorem{corollary}{Corollary}[section]
\theoremstyle{definition}
\newtheorem{observation}{Observation}[section]
\newtheorem{definition}{Definition}[section]
\newtheorem{propdef}{Proposition / Definition}[section]
\newtheorem{remark}{Remark}[section]
\newtheorem*{lemma*}{Lemma}
\newtheorem*{claim*}{Claim}

%
%\newtheorem{inneraxiom}{Axiom}
%\newenvironment{axiom}[1]
%{\renewcommand\theinneraxiom{#1}\inneraxiom}
%{\endinneraxiom}
%\newcommand{\cc}{\mathfrak{c}}


%\newcommand{\Hc}{\mathcal{H}}
%\newcommand{\Lan}{\mathcal{L}}
%
%\newcommand{\clist}{\mathfrak{c}_{1}, \cdots, \mathfrak{c}_m}
%\newcommand{\morph}[1]{\stackrel{#1}{\simeq}}
%\newcommand{\vlst}[2]{#1_1,\dots, #1_{#2}}
%\newcommand{\gnp}{G(n,\beta_1/n^{a_1-1}, \dots,\beta_l/n^{a_l-1})}
\newcommand{\Z}{\mathbb{Z}}
\newcommand{\CC}{\mathbb{C}}
\newcommand{\Q}{\mathbb{Q}}
\newcommand{\R}{\mathbb{R}}
\newcommand{\N}{\mathbb{N}}
\DeclarePairedDelimiter\floor{\lfloor}{\rfloor}
\newcommand{\Ln}{\lim\limits_{n\to \infty}}
\newcommand{\LN}{\lim\limits_{N\to \infty}}

\title{More proof Sketches Regarding the Closure of 
	Limiting Probabilities in Sparse Random Hyper-graphs}
\date{\today}
\author{Alberto Larrauri}


\begin{document}
	\maketitle 
\section{Preliminaries}
$G^d(N,p)$ denotes the binomial model of random $d$-uniform hyper-graphs. 
Consider $d\geq 2$ fixed for the rest of this writing. 

\begin{theorem} Let $c<(d-2)!$. Then, a.a.s all the components
of $G^d(N,p(N))$, where $p(N)\sim c/N^{d-1}$, are either trees or unicycles.
\end{theorem}
\begin{proof} See \cite{erdHos1960evolution} for $d=2$ and
	\cite{karonski2002phase} for the general case.
	\todo[inline]{Está todo hecho en el modelo uniforme, 
		pero la transferencia uniforme-$>$ binomial es ``sencilla''}
\end{proof}
We will call the \textbf{fragment} $F_N$ to the union of the unicyclic 
components in $G^d(N,p(N))$.


\begin{theorem} \label{thm:sizefragment}
	Let $c<(d-2)!$ and let $p(N)\sim c/N^{d-1}$. Let $\delta_N$ be the
	random variable that counts the vertices in $F_N$. 
	Then there is some constant $C$ such that 
	$\mathrm{E}[\delta_n]$ is smaller than $C$ for any $N$. 
	\todo[inline]{Este teorema lo he visto para grafos en el artículo de Erdos de 1960
	, y más completo en el libro de Karonski, por ejemplo. Para hypergrafos no lo he encontrado, pero igual es cosa de buscar mejor. Lo demuestro de todas formas.}
\end{theorem}
\begin{proof}
	Let $C_d(m)$ denote the number of connected labeled $d$-uniform
	hypergraphs with $m$ edges and $n:=(d-1)m$ vertices. It is satisfied 
	that for big values of $m$:
	\[C_d(m)\leq \frac{n^n}{e^{(m-n)}}\frac{1}{(d-2)!^m},\]
	where the constant hidden by the $O$-notation depends only on $d$.
	Let $X_{N,m}$ be the random variable that counts the number of
	unicyclic components in $G^d(N,c/N^{d-1})$ with exactly $m$ edges.
	Then we have $\delta_N =\sum_{m=2}^{N/(d-1)} (d-1)m \cdot X_{N,m}$.
	Also,
	\begin{align}
	\nonumber \mathrm{E}[X_{N,m}] & \leq 
	\binom{N}{n} p(N)^m
	(1-p(N))^{\binom{N}{d}-\binom{N-n}{d} - m}  \frac{n^n}{e^{(n-m)}}\frac{1}{(d-2)!^m}\\ 
%	& \leq \frac{N^n}{n!}e^{\frac{n(n-1)}{2N}}p(N)^m
%	e^{-p(N)\left[\binom{N}{d}-\binom{N-n}{d} - m\right]}
%	\frac{n^n}{e^{(m-n)}}\frac{1}{(d-2)!^m}\\
	&\nonumber \leq \frac{N^ne^n}{n^n}e^{-\frac{n(n-1)}{2N}}p(N)^m
	e^{-p(N)\left[\binom{N}{d}-\binom{N-n}{d} - m\right]}
	\frac{n^n}{e^{(n-m)}}\frac{1}{(d-2)!^m}\\
	&\label{eqn:bound} =\left(\frac{c}{(d-2)!}\right)^m e^{n-\frac{n(n-1)}{2N}
	-p(N)\left[\binom{N}{d}-\binom{N-n}{d} - m\right]+ 
	(m-n)} 
	\end{align}
	Operating on the exponent of $e$ in the last term:
	\begin{align*}
	&n-\frac{n(n-1)}{2N}
	-p(N)\left[\binom{N}{d}-\binom{N-n}{d} - m\right]+ 
	(m-n)\leq\\ 
	&- \frac{n(n-1)}{2N} - p(N)\left[\binom{N}{d}-\binom{N-n}{d}\right]
	+ p(N)m + m \leq \\
	& m\Big(-\frac{(d-1)}{2} - \frac{c}{(d-2)!} + p(N) + 1\Big)	\leq \\
	& m\Big(1 - \frac{c}{(d-2)!} \Big).	
	\end{align*}
	Thus, substituting in \cref{eqn:bound} we obtain
	\begin{equation}
	\label{eqn:expbound}
		\mathrm{E}[X_{N,m}]\leq \Bigg(\frac{c}{(d-2)!}
		e^{1-\frac{c}{(d-2)!}} \Bigg)^m
	\end{equation}
	One can easily check that $xe^{1-x}$ grows monotonously from $0$
	to $1$ as $x$ goes from $0$ to $1$. Thus, $\frac{c}{(d-2)!}
	e^{1-\frac{c}{(d-2)!}}<1$.\par
	By definition it is satisfied
	\[
	\mathrm{E}[\delta_N]=\sum_{m=2}^{N/(d-1)}	m \mathrm{E}[X_{N,m}],
	\]
	except for the case $d=2$, where the sum starts at $m=3$ instead of $m=2$.
	This is because there are no connected graphs with $2$ vertices and $2$ edges. 
	Because of \cref{eqn:expbound}, for sufficiently large values of $m$ the terms
	$m\mathrm{E}[X_{N,m}]$ are bounded by
	$m(\frac{c}{(d-2)!}e^{1-\frac{c}{(d-2)!}})^m$ uniformly for all values of $N$. 
	Otherwise, for small values of $m$, the $m\mathrm{E}[X_{N,m}]$ are bounded as
	well because
	\[
	\LN m\mathrm{E}[X_{N,m}]= \left(ce^{\frac{c}{(d-2)!}}\right)^m\frac{C_d(m)}{(m\cdot(d-1))!}
	\]
	In consequence, using the dominated converge theorem we can conclude that
	\[
	\LN \mathrm{E}[\delta_N]=\sum_{m=2}^{N/(d-1)}	m \mathrm{E}[X_{N,m}]
	\]
	exists and is not infinite. 
	\end{proof}
	
	\begin{theorem} \label{thm:expectedcycles}
		Let $c< (d-2)!$, and let $p(N)\sim c/N^{d-1}$. For any $k\geq 2$, let  $\gamma_{N,\geq k}$ be the random variable that counts how many 
		cycles with at least $k$ edges lie in $G^d(N,p(N))$. Then
		\[
		\LN \mathrm{E}[\gamma_{N,\geq k}]=\sum_{l=k}^{\infty} 
		\left(\frac{c}{(d-2)!}\right)^l \frac{1}{2l}
		\] 	
		In particular, if $\gamma_N$ is the random variable that counts the cycles
		in $G^d(N,p(N))$ then, if $d>2$:
		\[
		\LN \mathrm{E}[\gamma_{N}]	= 
		\begin{cases}
		\frac{c}{2(d-2)!}+ \ln\left(1-\frac{c}{2(d-2)!}\right), &  
		\text{ if } d>2\\
		\frac{c}{2}+ \frac{c^2}{4} 
		\ln\left(1-\frac{c}{2}\right), &
		\text{ if } d=2.
		\end{cases}
		\]
	\end{theorem}
	\todo[inline]{Esto está hecho en el artículo de Erdos para grafos, y para 
	hypergrafos no lo he visto.}
	\begin{proof}
		For $k\geq 2$ ($k\geq 3$ if $d=2$), let $\gamma_{N,k}$ be the random
		variable that counts
		the $k$-cycles that lie in $G^d(N,p(N))$.
		A simple computation yields
		$
		\mathrm{E}[\gamma_{N,k}]= \frac{(N)_{k(d-1)}}{2k}
		\left(\frac{c}{N^{d-1}}\right)^k,$
		and
		$
		\LN	\mathrm{E}[\gamma_{N,k}]= \frac{c^k}{2k}
		$.
		Not only this, but also
		$
		\mathrm{E}[\gamma_{N,k}]\leq \frac{c^k}{2k}$
		for all $N$. In consequence, applying the dominated convergence
		theorem we obtain
		\[ 
		\LN \sum_{l=k}^\infty \mathrm{E}[\gamma_{N,l}]=
		 \sum_{l=k}^\infty \frac{c^l}{2l}.
		\]
		Using that $\gamma_{N,\geq k}$ is the sum of all the $\gamma_{N,l}$
		for $l\geq k$ and the Taylor expansion of $ln(1-x)$ yields the desired
		results. 
	\end{proof}
	
	Given $0<c<(d-2)!$ we define the function $B_d(c)$ as 
	the limit as $N$ tends to infinity of the expected number of
	cycles in $G^d(N,c/N^{d-1})$. Because of our last theorem
	\[
	B_d(c)=\begin{cases}
	\frac{c}{2(d-2)!}+ \ln\left(1-\frac{c}{2(d-2)!}\right)
	,& \text{ if } d>2\\	
	\frac{c}{2}+ \frac{c^2}{4} \ln\left(1-\frac{c}{2}\right)
	, &\text{ if } d=2.
	\end{cases}
	\]
	
	\begin{corollary} \label{thm:limitchangeunicycles}
		Let $c<(d-2)!$ and $p(N)\sim c/N^{d-1}$. For each $k\in \N$ let 
		$Z_{N,k}$ be the random variable that counts the
		unicyclic components in $G^d(N,p(N))$ with exactly $k$ edges. 
		Then 
		\[
		\sum_{i=1}^\infty \LN \mathrm{E}\big[ Z_{N,i} \big]
		=  \LN \mathrm{E}\big[\sum_{i=1}^\infty Z_{N,i} \big]	
		= B_d(c).	
		\]
	\end{corollary}
	A simple application of the factorial moments method proves the following
	theorem:
	\begin{theorem} \label{thm:probcycles} 
		Let $k_1,\dots,k_j\geq 2$ ($\geq 3$ if $d=2$). Then, 
		as $N$ tends to infinity the $\gamma_{N,l_i}$'s converge 
		in distribution to independent Poisson variables
		with mean values $\lambda_i:=\frac{c^{l_i}}{2l_i}$ respectively.
		That is, for any $a_1,\dots,a_j\in \N$,
		\[
		\LN \mathrm{Pr}\Big(
		\bigwedge_{i=1}^r
		\gamma_{N,l_i}=a_i		
		\Big)= \prod_{i=1}^{j} e^{-\lambda_i}\frac{\lambda_i^{a_i}}{a_i!}
		\]
	\end{theorem}
  	\todo[inline]{Esto está hecho en el libro de Bollobás de random graphs. 
  	Para hypergrafos
  	no lo he encontrado, pero realmente es lo mismo. Si queréis lo hago.}
	
	
	Let $H$ be an unicycle and let $m=|E(H)|$, $n=|V(H)|=m(d-1)$. 
	Let $X_{N,H}$ be the random variable that counts
	the number of connected components in $G^d(N,p(N))$ isomorphic to $H$. Then
	\[ 
	\mathrm{E}[X_{N,H}]= \frac{(N)_n}{|Aut(H)|}
	p(N)^{m}(1-p(N))^{\binom{N}{d}-\binom{N-n}{d} - m}.
	\]
	And if $p(N)\sim c/N$,
	\[
	\LN \mathrm{E}[X_{N,H}]=  \frac{c^m}{|Aut(H)|}
	e^{\LN c\frac{\binom{N}{d}-\binom{N-n}{d} - m}{N^{d-1}}}
	=  \frac{c^m}{|Aut(H)|}
	e^{c\frac{m}{(d-2)!}}.
	\]	
	For convenience's sake we will often use the auxiliary variable
	$s=\frac{c}{(d-2)!}e^{\frac{c}{(d-2)!}}$. We can rewrite last 
	limit in terms of $s$ as:
	\[
	\LN \mathrm{E}[X_{N,H}]=(s)^m
	\frac{(d-2)!^m}{|Aut(H)|}.	
	\]
	Another application of the factorial moments method proves the next theorem:

\begin{theorem} \label{thm:probunicycliccomponents}
	Let $H_1,\dots, H_j$ be unicycles, and let $P(N)\sim c/N$. 
	Then, as $N$ tends to infinity, the $X_{N,H_i}$'s converge in 
	distribution to independent Poisson variables with means
	$\lambda_i=s^{|E(H_i)|}\frac{(d-2)!^{|E(H_i)|}}{|Aut(H_i)}$
	respectively. That is, for any $a_1,\dots, a_j\in \N$,
	\[
	\LN \mathrm{Pr}\Big(
	\bigwedge_{i=1}^r
	X_{N,H_i}=a_i		
	\Big)= \prod_{i=1}^{j} e^{-\lambda_i}\frac{\lambda_i^{a_i}}{a_i!}
	\]
\end{theorem} 

Next we show some stuff




\begin{theorem} 
	Let $0<c<(d-2)!$ and $p(N)\sim c/N$. For $i\in \N$ let 
	$\lambda_i=\left(\frac{c}{(d-2)!}\right)^i\frac{1}{2i!}$. 
	Let $A_N$ be the event that $G^d(n,p(N))$ contains no cycles. 
	Define $F_d:=F_d(c)$ as
	\[
	F_d(c)=\begin{cases}
	e^{\sum_{i=2}^{\infty} \lambda_i}=
	e^{\frac{c}{2(d-2)!}}\sqrt{1-\frac{c}{(d-2)!}} & \text{ if } d>2\\
	e^{\sum_{i=3}^{\infty} \lambda_i}=
	e^{\frac{c}{2}+\frac{c^2}{4}}\sqrt{1-c} & \text{ if } d=2.
	\end{cases}
	\]
	Then it is satisfied
	\[
	\LN \mathrm{Pr}\big(A_N \big)=F_d
	\]	
\end{theorem} 
\todo[inline]
{Esto está hecho para grafos en \cite{erdHos1960evolution}. }
\begin{proof}
	For each $k\in \N$ define $F_{d,k}:=F_{d,k}(c)$ as 
	\[
	F_{d,k}(c)=\begin{cases}
	e^{\sum_{i=2}^k -\lambda_i} 
	& \text{ if } d>2\\
	e^{\sum_{i=3}^k -\lambda_i} 
	& \text{ if } d=3.
	\end{cases}
	\]
	A simple computation using the Taylor expansion
	of $\ln(1-x)$ shows that $\lim\limits_{k\to \infty} F_{d,k}=F_d$.
	Fix an arbitrary $\epsilon>0$. We show that there exists a 
	constant $k$ satisfying
	\[
	\Big| \LN \mathrm{Pr}\big(A_N \big) - F_{d,j} \Big|\leq \epsilon 
	\text{ for any } j\geq k.
	\]
	For each $k\in N$ let $A_{N,k}$ be the event that $G^d(N,p(N))$ contains 
	no cycle with length at most $k$. Using \cref{thm:probcycles}, we obtain 
	\[
	\LN \mathrm{Pr}\big(A_{N,k}\big)=F_{d,k}.\]
	For any $k\in \N$ let $\gamma_{N,\geq k}$ be the random variable that
	counts the number of cycles in $G^d(N,p(N))$ with length at least $k$. 
	Then, using \cref{thm:expectedcycles} we obtain that 
	$\lim\limits_{k\to \infty}	\LN \mathrm{E}[\gamma_{N,\geq k}]=0$. \\
	Fix $k$ such that for any $j\geq k$
	\[
	\Big|\LN \mathrm{E}[\gamma_{N,\geq k}]\Big| \leq \epsilon.
	\]
	Notice that for any $j$, "$A_N$ is true and $A_{N,j}$ is false"
	if and only if $\mathrm{Pr}\big(\gamma_{N,\geq j} \geq 1\big)$. 
	Also, by Markov inequality, 
	$\mathrm{Pr}\big(\gamma_{N,\geq j} \geq 1\big)
	\leq \mathrm{E}\big[\gamma_{N,\geq j}\big]$.
	In consequence, for any $j\geq k$
	\[
	\LN \Big|\mathrm{Pr}\big(A_N\big)- \mathrm{Pr}\big(A_{N,j}\big)\Big|
	\leq \epsilon.
	\]
	In consequence, for any $j\geq k$
	\[ \Big| \LN \mathrm{Pr}\big(A_N \big) - F_{d,j} \Big|
	\leq  \Big| \LN \mathrm{Pr}\big(A_{N,j} \big) - F_{d,j} \Big| +
	\epsilon = \epsilon,	
	\]
	and we are finished. 
\end{proof}


\begin{theorem} 
	Let $0<c<(d-2)!$, and $p(N)\sim c/N^{d-1}$.
	Let $H$ be an hypergraph whose components are all 
	unicycles, and let  $F_d:=F_d(c)$ be as in last theorem. 
	Then
	\[
	\LN \mathrm{Pr}\Big(F_N\simeq H\Big)=F_d(c)\left(
	\frac{c}{(d-2)!}e^{\frac{c}{(d-2)!}}\right)^{e(H)} \frac{(d-2)!^{e(H)}}{|Aut(H)|}
	\]
\end{theorem}

\begin{proof}
	Fix such $H$.
	Let $U_1,U_2,\dots, U_i,\dots$ be an enumeration of all unicycles in a way
	such that $e(U_i)\leq e(U_j)$ if $i\leq j$.  For each $i\in \N$ let $a_i$
	be the number of connected components of $H$ that are isomorphic to $U_i$,
	and let $X_{N,i}$ be the random variable that counts the number of connected 
	components in $G^d(N,P(N))$ that are isomorphic to $U_i$. Clearly,
	$F_N\simeq H$ if and only if $X_{N,i}=a_i$ for all $i$.  Thus,
	\[
	\LN \mathrm{Pr}\big(F_N\simeq H\big)= 
	\LN \mathrm{Pr}\big( \bigwedge_{i=1}^\infty X_{N,i}=a_i\big).
	\]
	First, we are going to show that 
	\[
	\LN \mathrm{Pr}\big( \bigwedge_{i=1}^\infty X_{N,i}=a_i\big)=
	\lim\limits_{j\to \infty}
	\LN \mathrm{Pr}\big( \bigwedge_{i=1}^j X_{N,i}=a_i\big).
	\]
	Fix $\epsilon > 0$ an arbitrarily small real constant. We need
	to prove that there exists some $j_0\in \N$ satisfying
	\[
	\left|\LN \mathrm{Pr}\big( \bigwedge_{i=1}^j X_{N,i}=a_i\big)
	- \lim\limits_{j\to \infty}
	\LN \mathrm{Pr}\big( \bigwedge_{i=1}^j X_{N,i}=a_i\big)
	\right|
	\leq \epsilon
	\text{ for all } j\geq j_0.
	\]
	For each $k \in \N$ let $Y_{N,k}$ be the random variable
	that counts the uni-cyclic connected components of $G^d(N,p(N))$
	with exactly $k$ edges.	By \cref{thm:sizefragment}
	we have that for some $k_0\in \N$
	\[ 
	\LN \sum_{l=k_0}^{\infty }\mathrm{E}\big[ Y_{N,l} \big] \leq 
	\epsilon
	\]
	Let $k_1$ be the maximum number of edges in a connected component of 
	$H$, and let $k = \max(k_0, k_1+1)$. Finally, fix $j_0$ such that 
	$e(U_j)>k_1$ for any $j\geq j_0$. \par
	Given any $j\geq j_0$, $F_N \simeq H$ if and only if 
	\[
	\big(\bigwedge_{i=1}^{j} X_{N,i}=a_i\big) \wedge
	\big(\sum_{l=k}^\infty Y_{N,l}=0  \big) 
	\]
	Using Markov inequality we get 
	\[
	\LN \mathrm{Pr}\big( \sum_{l=k}^\infty Y_{N,l}\geq 1  \big) \leq \epsilon
	\]
	Thus,
	\[
	\left|\LN \mathrm{Pr}\big( \bigwedge_{i=1}^j X_{N,i}=a_i\big)
	- \lim\limits_{j\to \infty}
	\LN \mathrm{Pr}\big( \bigwedge_{i=1}^j X_{N,i}=a_i\big)
	\right|
	\leq \epsilon,
	\]
	as we wanted to prove. 	\par
	Using \cref{thm:probunicycliccomponents} we get
	\[
	\lim\limits_{j\to \infty}
	\LN \mathrm{Pr}\big( \bigwedge_{i=1}^j X_{N,i}=a_i\big)=
	\prod_{i=1}^\infty e^{-\lambda_i} \frac{\lambda_i^{a_i}}{a_i!},
	\]
	where $\lambda_i=s^{e(U_i)}\frac{(d-2)!^{e(U_i)}}{|Aut(U_i)|}= 
	\LN \mathrm{E}[X_{N,i}]$, and $s=s(c)$ is defined as in the statement 
	of the theorem. Using \cref{thm:limitchangeunicycles} we obtain
	\[
	\sum_{i=1}^{\infty} \lambda_i = B_d(c),
	\]
	and in consequence 
	\[
	\prod_{i=1}^\infty e^{-\lambda_i}= e^{-B_d(c)}=A_d(c).
	\]
	The following identities hold:
	\[
	\sum_{i=1}^\infty e(U_i)a_i = e(H) \qquad 
	\prod_{i=1}^{\infty} |Aut(U_i)^{a_i} a_i!= |Aut(H)|.
	\]
	As a consequence we get
	\[
	\prod_{i=1}^\infty \frac{\lambda_i^{a_i}}{a_i!}=
	\prod_{i=1}^\infty (s(d-2)!)^{e(U_i)a_i} 
	\frac{1}{|aut(U_i)|^{a_i}a_i!}=
	s^{e(H)}\frac{(d-2)!^{e(H)}}{|Aut(H)|},
	\]
	and the theorem follows. 
\end{proof}

Let us denote by $\mathcal{U}$ the class of hypergraphs
whose connected components are unicycles. For the case $d=2$ 
the asymptotic distribution of the fragment $F_N$ coincides with
the Boltzmann-Poisson distribution of random graphs from $\mathcal{U}$
described in \cite{mcdiarmid2009random}.
\todo[inline]{Tobias de una forma un poco críptica da a entender que
	usar el teorema 1.3 de \cite{mcdiarmid2009random} sirve para algo. Yo no he sabido cómo
	o para qué.}
\par
For completeness sake we give an argument of why the asymptotic distribution
of $F_N$ is indeed a probability distribution.  



\begin{theorem}
	Let $0<c<(d-2)!$ and let $p(N)\sim c/N^{d-1}$. 
	Let $H_1, \dots H_k, \dots$ be an enumeration of all hypergraphs
	in $\mathcal{U}$. For each $i\in \N$ 
	denote by $p_i$ the limit $\LN \mathrm{Pr}(F_N \simeq H_i)$. Then
	\[
	\sum_{i=1}^\infty p_i = 1.	
	\]
\end{theorem}
\begin{proof}
	Let $\epsilon>0$ be an arbitrarily small real constant. We show that 
	there exists some $j_0\in \N$ such that
	\[
	1- \sum_{i=1}^j p_i \leq \epsilon \quad \text{ for all } j\geq j_0.	
	\]	
	Let $m=\LN \mathrm{E}[e(F_N)]$. Notice that $m$ exists by \cref{thm:sizefragment}.
	Define $M=m/\epsilon$. Then 
	\[
	\LN \mathrm{Pr}\big(e(F_N)\geq M\big)\leq \epsilon.
	\]
	Let $j_0$ be such that $e(H_i)\geq M$ for all $i\geq j_0$. 
	Then, given any $j\geq j_0$, 
	\[
	1 - \sum_{i=1}^j p_i = \LN \mathrm{Pr}\big(
	\bigwedge_{i=1}^j F_N \not\simeq H_i\big) \leq
	\LN \mathrm{Pr}\big(
	e(F_N) \geq M \big)\leq \epsilon, 	
	\]	
	and the result follows. 
\end{proof}


\begin{theorem}
	Let $0<c<(d-2)!$, and $p(N)\sim c/N^{d-1}$. Let 
	$H_1,\dots, H_i,\dots$ be an enumeration of all
	hypergraphs in $\mathcal{U}$ and for each $i\in \N$ 
	let $p_i=\LN \mathrm{Pr}(F_N\simeq H_i)$. 
	Consider the sets 
	\[
	L_c:=\{
	\LN \mathrm{Pr}(G^d(N,p(N)) \text{ satisfies } P) \, | \,
	P \text{ FO property} \},
	\] 
	and 
	\[
	S_c:=\{
	\sum_{H_i\in T} p_i \, | \, T \subset \mathcal{U} \,
	\}.	
	\]
	Then it is satisfied that $\overline{L_c}=S_c$. 		
\end{theorem}
\begin{proof}
	We will prove the statement by showing both 
	$S_c\subseteq \overline{L_c}$ and
	$\overline{L_c}\subseteq S_c$. We begin with 
	$S_c\subseteq \overline{L_c}$. 
	Let $T\subset \mathcal{U}$, and let $\epsilon>0$ be
	an arbitrarily small real number. We show that 
	there exists a first order property $P$ such that
	\[ \Big| \LN \mathrm{Pr}\big( G^d(N,p(N)) \text{ satisfies } P \big) 
	- \sum_{H_i\in T} p_i \Big| \leq \epsilon \].
	As 
	
	
	
	For each $k\in \N$ let $F_{N,k}$ be the union of connected components
	in $G^d(N, p(N))$ that are unicycles with no more than $k$ edges, and let
	$\mathcal{U}_k$ be the class of hypergraphs whose connected components
	are unicycles with at most $k$ edges. 
	Given any $H\in \mathcal{U_k}$, the property $F_{N,k}\simeq H$ is 
	expressible in FO logic. 
	
\end{proof}



\bibliography{biblio}
\bibliographystyle{unsrt}
\end{document}