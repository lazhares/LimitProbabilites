\documentclass[11pt,notitlepage,a4paper]{article}

\usepackage[left=2cm,right=2cm,top=2cm,bottom=2cm]{geometry}
\usepackage{graphicx}
%%BeginIpePreamble
\usepackage{amssymb,mathtools, amsmath, amsfonts, amsthm}
%%EndIpePreamble
%\usepackage{color}
\usepackage{float}
\usepackage{hyperref}
\usepackage{enumerate}
\usepackage{enumitem}
\usepackage{chngcntr}
\usepackage{cleveref}
\usepackage{pdfpages}
\usepackage{caption,subcaption,float}
\usepackage[utf8]{inputenc}

\counterwithout{equation}{section}



\newlength{\margen}
\setlength{\margen}{\paperwidth}
\addtolength{\margen}{-\textwidth}
\addtolength{\skip\footins}{0.7 cm}
\setlength{\margen}{0.5\margen}
\addtolength{\margen}{-1in}
\setlength{\oddsidemargin}{\margen}
\setlength{\evensidemargin}{\margen}
\setlength{\abovedisplayskip}{3pt}
\setlength{\belowdisplayskip}{3pt}
%%%% Small setup %%%%
\hypersetup{
	colorlinks=false,
	pdfborder={1 1 0.0005},
}
\setlength{\parskip}{0.2cm}
%%%%%%%%%%%%%%%
%\usepackage{tikz-cd}
%\usetikzlibrary{cd}
\usepackage[english]{babel}
\usepackage{todonotes}
%\usepackage{cleveref}
%\usepackage{caption}
%\usepackage{subcaption}
%\usepackage{bbding}
%\usepackage{tcolorbox}
%\usepackage{natbib}
%
%\DeclareMathOperator{\incl}{incl}

\newtheorem{proposition}{Proposition}[section]
\newtheorem{fact}{Fact}[section]
\newtheorem{theorem}{Theorem}[section]
\newtheorem{lemma}{Lemma}[section]
\newtheorem{corollary}{Corollary}[section]
\theoremstyle{definition}
\newtheorem{observation}{Observation}[section]
\newtheorem{definition}{Definition}[section]
\newtheorem{propdef}{Proposition / Definition}[section]
\newtheorem{remark}{Remark}[section]
\newtheorem*{lemma*}{Lemma}
\newtheorem*{claim*}{Claim}

%
%\newtheorem{inneraxiom}{Axiom}
%\newenvironment{axiom}[1]
%{\renewcommand\theinneraxiom{#1}\inneraxiom}
%{\endinneraxiom}
%\newcommand{\cc}{\mathfrak{c}}


%\newcommand{\Hc}{\mathcal{H}}
%\newcommand{\Lan}{\mathcal{L}}
%
%\newcommand{\clist}{\mathfrak{c}_{1}, \cdots, \mathfrak{c}_m}
%\newcommand{\morph}[1]{\stackrel{#1}{\simeq}}
%\newcommand{\vlst}[2]{#1_1,\dots, #1_{#2}}
%\newcommand{\gnp}{G(n,\beta_1/n^{a_1-1}, \dots,\beta_l/n^{a_l-1})}
\newcommand{\Z}{\mathbb{Z}}
\newcommand{\CC}{\mathbb{C}}
\newcommand{\Q}{\mathbb{Q}}
\newcommand{\R}{\mathbb{R}}
\newcommand{\N}{\mathbb{N}}
\DeclarePairedDelimiter\floor{\lfloor}{\rfloor}
\newcommand{\Ln}{\lim\limits_{n\to \infty}}
\newcommand{\LN}{\lim\limits_{N\to \infty}}

\title{More proof Sketches Regarding the Closure of 
	Limiting Probabilities in Sparse Random Hyper-graphs}
\date{\today}
\author{Alberto Larrauri}


\begin{document}
	\maketitle 
\section*{Preliminaries}
$G^d(N,p)$ denotes the binomial model of random $d$-uniform hypergraphs. 
We will consider $d\geq 2$ fixed for the rest of this writing. 
We will deal with the case where $p=p(N)= c/N^{d-1}$ for some real number 
$c> 0$. For each $N\in \N$, we will use $G_N:=G_N(c)$ to denote a random sample from $G^d(N,p)$. We will refer to $d$-uniform hypergraphs simply as hypergraphs.
\todo[inline]{Si no me equivoco todo lo de aquí sirve también para el caso
en el que $p(N)\sim c/n^{d-1}$, no hace falta que sean iguales. Pongo igualdad
porque es más fácil de escribir.}

We call $k$-cycles to cycles with exactly $k$ edges.\par

For each $n,m\in \N$ with $m\leq n$ we will
denote the product $n(n-1)\cdots (n-m+1)$ by 
$(n)_m$. 

\section*{Theorems}
\begin{theorem} \label{thm:subcritical}
Let $0<c<(d-2)!$. Then, a.a.s all the components
of $G_N$, are either trees or unicycles.
\end{theorem}
\begin{proof} See \cite{erdHos1960evolution} for $d=2$ and
	\cite{karonski2002phase} for the general case.
	\todo[inline]{Está todo hecho en el modelo uniforme, 
		pero la transferencia uniforme-$>$ binomial es ``sencilla''}
\end{proof}
\noindent\rule{2cm}{0.4pt}\par
Given an hypergraph $G$, we will call the \textbf{fragment} $F(G)$
of $G$ to the union of its unicyclic components. We will abbreviate
$F(G_N)$ as $F_N$.


\begin{theorem} \label{thm:sizefragment}
	Let $0<c<(d-2)!$. For each $i\in \N$ let
	$X_{N,i}$ be the random variable that
	counts the number of unicyclic components 
	in $G_N$ that contain exactly $i$ edges.
	Then $\LN \mathrm{E}[e(F_N)]$  
	exists and is a finite quantity. Furthermore,
	\[
	\LN \mathrm{E}[e(F_N)]=\sum_{i=1}^\infty i \LN X_{N,i}.
	\]
	\todo[inline]{Este teorema lo he visto para grafos en el artículo de Erdos de 1960
	, y más completo en el libro de Karonski, por ejemplo. Para hypergrafos no lo he encontrado, pero igual es cosa de buscar mejor. Lo demuestro de todas formas.}
\end{theorem}
\begin{proof}
	Let $C_d(m)$ denote the number of connected labeled $d$-uniform
	hypergraphs with $m$ edges and $n:=(d-1)m$ vertices. 
	By definition $e(F_N) =\sum_{i=1}^{\infty} i \cdot X_{N,i}$, and
	in consequence 
	\[
	\LN \mathrm{E}[e(F_N)]=\LN \sum_{i=1}^\infty i 
	\mathrm{E}\big[ X_{N,i}\big].
	\]
	To prove the statement we have to show that we can 
	exchange the limit and summation from the RHS of last equality.
	For that we will use the dominated convergence theorem. 
	First, notice that for each $i\in \N$,
	\begin{align*}
	\LN	\mathrm{E}\big[X_{N,i}\big]&=
	\LN \binom{N}{(d-1)i}\cdot\left(\frac{c}{N^{d-1}}\right)^i
	\cdot\left(1- \frac{c}{N^{d-1}}\right)^{\binom{N}{d}-
	\binom{N-(d-1)i}{d} - i}\cdot C_d(i)\\
	&=\left(c e^{-\frac{c}{(d-2)!}}\right)^i\cdot 
	\frac{C_d(i)}{((d-1)i)!}.
	\end{align*}
	Here we have used that
	for any fixed $i$ $\binom{N}{d}-
	\binom{N-(d-1)i}{d} - i\sim i N^{(d-1)}/(d-2)!$ as
	$N$ tends to infinity, and it follows that
	\[
	\Ln \left(1- \frac{c}{N^{d-1}}\right)^{\binom{N}{d-1}-
	\binom{N-(d-1)i}{d-1} - i} =
	e^{-\frac{ic}{(d-2)!}}.
	\]
	Now we have to dominate the terms $\mathrm{E}[X_{N,i}]$
	by some sequence $a_i$ such that $\sum_{i=1}^\infty ia_i\leq\infty$.\par
	It has been 
	proven (see \cite{karonski2002phase}) 
	that for big values of $i$
	\[C_d(i)\leq \frac{((d-1)i)^{(d-1)i}}{e^{((d-2)i)}}\frac{1}{(d-2)!^i},\]
	In consequence, if we name $j=(d-1)i$, for sufficiently large $i$ we
	obtain
	\begin{align}
	\nonumber \mathrm{E}[X_{N,i}] & \leq 
	\binom{N}{j}\cdot p(N)^{i}\cdot
	(1-p(N))^{\binom{N}{d}-\binom{N-j}{d} - i}
	\cdot\frac{{j}^{j}}{e^{((d-2)i)}}\frac{1}{(d-2)!^i}\\ 
%	& \leq \frac{N^n}{n!}e^{\frac{n(n-1)}{2N}}p(N)^m
%	e^{-p(N)\left[\binom{N}{d}-\binom{N-n}{d} - m\right]}
%	\frac{n^n}{e^{(m-n)}}\frac{1}{(d-2)!^m}\\
	&\nonumber \leq \frac{N^{j}e^{j}}{j^j}\cdot 
	p(N)^i \cdot
	e^{-p(N)\left[\binom{N}{d}-\binom{N-j}{d} - i\right]}
	\cdot \frac{j^j}{e^{(d-2)i}}\frac{1}{(d-2)!^i}\\
	&\label{eqn:bound} =\left(\frac{c}{(d-2)!}\right)^i 
	\cdot
	e^{j-p(N)\left[\binom{N}{d}-\binom{N-j}{d} - i\right]- 
	(d-2)i} 
	\end{align}
	Here we have used the bounds
	$1/j!\leq e^j/j^j$, $\quad (N)_j\leq
	N^j$, and 
	$(1-p(N))\leq e^{-p(N)}$.
	Operating on the exponent of $e$ in the last term
	we get, for sufficiently large $N$ uniformly in 
	$i$,
	\begin{align*}
	&j
	-p(N)\left[\binom{N}{d}-\binom{N-j}{d} - i\right]- 
	(d-2)i \leq \\ 
	&- p(N)\left[\binom{N}{d}-\binom{N-j}{d}\right]
	+ p(N)i + i \leq \\
	& i\Big(1 - \frac{c}{(d-2)!}\Big) + c
	\end{align*}
	Here we have used that
	$\binom{N}{d}-\binom{N-j}{d}\leq N^{d-1}j/(d-1)!$,
	and also $p(N)i\leq c$ because we can suppose that $i<<N^{d-1}$ (otherwise 
	$\mathrm{E}[X_{N,i}]=0$).
	Thus, substituting at the end of 
	\cref{eqn:bound} we obtain
	\begin{equation}
	\label{eqn:expbound}
		\mathrm{E}[X_{N,i}]\leq \Bigg(\frac{c}{(d-2)!}
		e^{1-\frac{c}{(d-2)!}} \Bigg)^i \cdot e^c,
	\end{equation}
	for sufficiently large $i$ and large $N$ uniformly
	in $i$. 
	One can easily check that $xe^{1-x}$ grows monotonously from $0$
	to $1$ as $x$ goes from $0$ to $1$. Thus, $\frac{c}{(d-2)!}
	e^{1-\frac{c}{(d-2)!}}<1$, and
	\[
	\sum_{i=1}^\infty i\cdot\left(\frac{c}{(d-2)!}
	e^{1-\frac{c}{(d-2)!}} \right)^i\cdot e^c < \infty.
	\]
	Hence, using \cref{eqn:expbound} we can
	apply the dominated convergence theorem
	obtaining
	\[
	\LN \sum_{i=1}^\infty i 
	\mathrm{E}\big[ X_{N,i}\big]
	= \sum_{i=1}^\infty i \LN \mathrm{E}\big[ X_{N,i}\big]
	,
	\]
	and both expressions are finite quantities, proving
	the theorem. 
	\end{proof}
	\noindent\rule{2cm}{0.4pt}\par
	
	\begin{theorem} \label{thm:expectedcycles}
		Let $0< c< (d-2)!$. For any $k\in \N$, 
		let $Y_{N,k}$ be the random
		variable that counts
		the $k$-cycles that lie in $G_N$,
		and let $Y_{N,\geq k}= \sum_{l=1}^k Y_{N,l}$.
		Then
		\[
		\LN \mathrm{E}\big[Y_{N,\geq k}\big]=
		\sum_{l=k}^{\infty} \LN \mathrm{E}
		\big[Y_{N,l}\big]=
		\sum_{l=k}^{\infty} 
		\left(\frac{c}{(d-2)!}\right)^l \frac{1}{2l}
		\] 	
		In particular, if $Y_N$ is the random variable that counts the cycles
		in $G_N$ then:
		\[
		\LN \mathrm{E}[Y_{N}]	= 
		\begin{cases}
		\frac{c}{2(d-2)!}+ \ln\left(1-\frac{c}{2(d-2)!}\right), &  
		\text{ if } d>2\\
		\frac{c}{2}+ \frac{c^2}{4} 
		\ln\left(1-\frac{c}{2}\right), &
		\text{ if } d=2.
		\end{cases}
		\]
	\end{theorem}
	\todo[inline]{Esto está hecho en el artículo \cite{erdHos1960evolution}
	para grafos, y para hypergrafos no lo he visto.}
	\begin{proof}
	
		A simple computation yields that for 
		any $k\geq 2$ ($k\geq 3$ if $d=2$),
		$
		\mathrm{E}[Y_{N,k}]= 
		\frac{(N)_{k(d-1)}}{2k}
		\left(\frac{c}{N^{d-1}(d-2)!}\right)^k,$
		and
		$
		\LN	\mathrm{E}[Y_{N,k}]= 
		\left(\frac{c}{(d-2)!}\right)^k \frac{1}{2k}
		$.
		Also,
		$
		\mathrm{E}[Y_{N,k}]\leq \left(\frac{c}{(d-2)!}\right)^k \frac{1}{2k}$
		for all $N$. In consequence, applying the dominated convergence
		theorem we obtain
		\[ 
		\LN \sum_{l=k}^\infty \mathrm{E}[Y_{N,l}]=
		 \sum_{l=k}^\infty \left(\frac{c}{(d-2)!}\right)^l \frac{1}{2l} \quad 
		 \text{ for any } k\in \N.
		\]
		And using the Taylor expansion of $ln(1-x)$ we obtain
		the desired	result. 
	\end{proof}
	\noindent\rule{2cm}{0.4pt}\par

	
	
	A simple application of the factorial moments method proves the following
	theorem:
	\begin{theorem} \label{thm:probcycles} 
		For each $k\in \N$, let $Y_{N,k}$ be the random variable
		that counts the number of $k$-cycles in $G_N$.
		Let $k_1,\dots,k_j\geq 2$ ($\geq 3$ if $d=2$). Then, 
		as $N$ tends to infinity the $Y_{N,l_i}$'s converge 
		in distribution to independent Poisson variables
		with mean values $\lambda_i:=\frac{c^{l_i}}{2l_i}$ respectively.
		That is, for any $a_1,\dots,a_j\in \N$,
		\[
		\LN \mathrm{Pr}\Big(
		\bigwedge_{i=1}^r
		Y_{N,l_i}=a_i		
		\Big)= \prod_{i=1}^{j} e^{-\lambda_i}\frac{\lambda_i^{a_i}}{a_i!}
		\]
	\end{theorem}
	\todo[inline]{Esto está hecho en el libro de Bollobás de random graphs. 
		Para hypergrafos
		no lo he encontrado, pero realmente es lo mismo. Si queréis lo hago.}
		
	\noindent\rule{2cm}{0.4pt}\par
		
	Given $0<c<(d-2)!$ we define the function $B_d(c)$ as 
	the limit as $N$ tends to infinity of the expected number of
	cycles in $G_N(c)$. Because of \cref{thm:expectedcycles}
	\[
	B_d(c)=\begin{cases}
	\frac{c}{2(d-2)!}+ \ln\left(1-\frac{c}{2(d-2)!}\right)
	,& \text{ if } d>2\\	
	\frac{c}{2}+ \frac{c^2}{4} \ln\left(1-\frac{c}{2}\right)
	, &\text{ if } d=2.
	\end{cases}
	\]	
		
	\begin{corollary} \label{thm:limitchangeunicycles}
		Let $0<c<(d-2)!$. For each $k\in \N$ let 
		$X_{N,k}$ be the random variable that counts the
		unicyclic components in $G_N$ with exactly $k$ edges. 
		Then 
		\[
	 	B_d(c)=\LN \mathrm{E}\big[\sum_{i=1}^\infty X_{N,i} \big]	
		= \sum_{i=1}^\infty \LN \mathrm{E}\big[ X_{N,i} \big].
		\]
	\end{corollary}
	\begin{proof}
	We begin by showing that
	\begin{equation} \label{eqn:cyclesunicycles}
		B_d(c)=\LN \mathrm{E}\big[\sum_{i=1}^\infty X_{N,k}].
	\end{equation}
	Let $X_N=\sum_{i=1}^{\infty} X_{N,k}$ be the random variable
	that counts the unicyclic components in $G_N$. Let $Y_N$ count
	the cycles in $G_N$ and $Z_N$ count the cycles in $G_N$ that 
	belong to some non-unicyclic component. It is satisfied that 
	$X_N=Y_N-Z_N$. One can show that $\LN \mathrm{E}[Z_N]=0$ using that 
	(1) $\LN \mathrm{E}[Y_N]=B_d(c)$
	by \cref{thm:expectedcycles}, (2) $Z_N\leq Y_N$ for all $N\in \N$, and
	(3)	$\LN \mathrm{Pr}(Z_N>0)=0$ because of \cref{thm:subcritical}. 
	In consequence,
	\[
	 \LN \mathrm{E}\big[X_N\big]=\LN \mathrm{E}\big[Y_N\big]=B_d(c),
	\]
	as we wanted. 
	\par
%	For each $k$, let $Y_{N,k}$  
%	be the random variables that counts the 
%	$k$-cycles in $G_N$ and let $Z_{N,k}$ count how many of 
%	those $k$-cycles belong to a non-unicyclic component.
%	Define $Y_N=\sum_{i=1}^{\infty} Y_{N,i}$ and 
%	$Z_N=\sum_{i=1}^{\infty} Z_{N,i}$.\par
%	By definition $B_d(c)=\LN \mathrm{E}[Y_N]$, and
%	$B_d(c)-\LN \mathrm{E}[X_N]=\LN \mathrm{E}[Z_N]$.
%	Let $\epsilon>0$ be an arbitrarily small real number. 
%	We will show that $\LN \mathrm{E}[Z_N]\leq\epsilon$, and 
%	this will prove \cref{eqn:cyclesunicycles}.
%	Using \cref{thm:expectedcycles} one can prove that
%	there exists some $j$ such that $\LN \mathrm{E}
%	[\sum_{i=j}^{\infty} Y_{N,i}]\leq \epsilon/2$.
%	Using that $Z_{N,i}\leq Y_{N,i}$ for all $N$ and $i$
%	we obtain that 
%	\[
%	\LN \mathrm{E}\big[ \sum_{i=1}^\infty Z_{N,i}\big]
%	\leq \LN \mathrm{E}\big[ \sum_{i=1}^{j-1} Z_{N,i}\big] +
%	\epsilon/2
%	\leq
%	\LN \mathrm{E}\big[ \sum_{i=1}^{j-1} Y_{N,i}\big] +
%	\epsilon/2.
%	\]
%	Let $Y_{N,<j}=\sum_{i=1}^{j-1} Y_{N,i}$. 
%	Using $\cref{thm:probcycles}$ and the fact that
%	the sum of Poisson variables is a Poisson variable 
%	whose mean is the sum of the means of the initial variables
%	we get that $Y_{N,<j}$ converge in distribution to 
%	a Poisson variable with mean 
%	
	The equality $\LN \mathrm{E}\big[\sum_{i=1}^\infty X_{N,i} \big]	
	= \sum_{i=1}^\infty \LN \mathrm{E}\big[ X_{N,i} \big]$
	follows from applying the dominated converge theorem as in 
	\cref{thm:sizefragment}.
	\end{proof}
		\noindent\rule{2cm}{0.4pt}\par
	
	Let $H$ be an unicycle and let $i=e(H)$, $j=v(H)=i(d-1)$. 
	Let $X_{N,H}$ be the random variable that counts
	the number of connected components in $G_N$ isomorphic to $H$. Then
	\[ 
	\mathrm{E}[X_{N,H}]= \frac{(N)_i}{|Aut(H)|}
	p(N)^{i}(1-p(N))^{\binom{N}{d}-\binom{N-j}{d} - i}.
	\]
	And substituting $p(N)=c/N^{d-1}$,
	\[
	\LN \mathrm{E}[X_{N,H}]=  \frac{c^i}{|Aut(H)|}
	e^{\LN c\frac{\binom{N}{d}-\binom{N-j}{d} - i}{N^{d-1}}}
	=  \frac{c^i}{|Aut(H)|}
	e^{c\frac{i}{(d-2)!}}.
	\]	
	For convenience's sake we will sometimes use the auxiliary variable
	$s=\frac{c}{(d-2)!}e^{\frac{c}{(d-2)!}}$. We can rewrite last 
	limit in terms of $s$ as:
	\[
	\LN \mathrm{E}[X_{N,H}]=(s)^i
	\frac{(d-2)!^i}{|Aut(H)|}.	
	\]
	Another application of the factorial moments method proves the next theorem:

\begin{theorem} \label{thm:probunicycliccomponents}
	Let $H_1,\dots, H_j$ be unicycles. 
	Then, as $N$ tends to infinity, the $X_{N,H_i}$'s converge in 
	distribution to independent Poisson variables with means
	$\lambda_i=s^{e(H_i)}\frac{(d-2)!^{e(H_i)}}{|Aut(H_i)|}$
	respectively. That is, for any $a_1,\dots, a_j\in \N$,
	\[
	\LN \mathrm{Pr}\Big(
	\bigwedge_{i=1}^r
	X_{N,H_i}=a_i		
	\Big)= \prod_{i=1}^{j} e^{-\lambda_i}\frac{\lambda_i^{a_i}}{a_i!}
	\]
\end{theorem} 

\noindent\rule{2cm}{0.4pt}\par


\begin{theorem} 
	Let $0<c<(d-2)!$. For $i\in \N$ let 
	$\lambda_i:=\lambda_i(c)=\left(\frac{c}{(d-2)!}\right)^i\frac{1}{2i!}$. 
	Let $A_N$ be the event that $G^d(n,p(N))$ contains no cycles. 
	Define $F_d:=F_d(c)$ as
	\[
	F_d(c)=\begin{cases}
	e^{\sum_{i=2}^{\infty} \lambda_i}=
	e^{\frac{c}{2(d-2)!}}\sqrt{1-\frac{c}{(d-2)!}} & \text{ if } d>2\\
	e^{\sum_{i=3}^{\infty} \lambda_i}=
	e^{\frac{c}{2}+\frac{c^2}{4}}\sqrt{1-c} & \text{ if } d=2.
	\end{cases}
	\]
	Then it is satisfied
	\[
	\LN \mathrm{Pr}\big(A_N \big)=F_d
	\]	
\end{theorem} 
\todo[inline]
{Esto está hecho para grafos en \cite{erdHos1960evolution}. }
\begin{proof}
	For each $k\in \N$ define $F_{d,k}:=F_{d,k}(c)$ as 
	\[
	F_{d,k}=\begin{cases}
	e^{\sum_{i=2}^k -\lambda_i} 
	& \text{ if } d>2\\
	e^{\sum_{i=3}^k -\lambda_i} 
	& \text{ if } d=3.
	\end{cases}
	\]
	A simple computation using the Taylor expansion
	of $\ln(1-x)$ shows that $\lim\limits_{k\to \infty} F_{d,k}=F_d$.
	Fix an arbitrary $\epsilon>0$. We show that there exists a 
	constant $k$ satisfying
	\[
	\Big| \LN \mathrm{Pr}\big(A_N \big) - F_{d,j} \Big|\leq \epsilon 
	\text{ for any } j\geq k.
	\]
	For each $k\in N$ let $A_{N,k}$ be the event that $G^d(N,p(N))$ contains 
	no cycles with length at most $k$. Using \cref{thm:probcycles}, we obtain 
	\[
	\LN \mathrm{Pr}\big(A_{N,k}\big)=F_{d,k}.\]
	For any $k\in \N$ let $Y_{N,\geq k}$ be the random variable that
	counts the number of cycles in $G^d(N,p(N))$ with length at least $k$. 
	Then, using \cref{thm:expectedcycles} we obtain that 
	$\lim\limits_{k\to \infty}	\LN \mathrm{E}[Y_{N,\geq k}]=0$. \\
	Fix $k$ such that for any $j\geq k$
	\[
	\Big|\LN \mathrm{E}[Y_{N,\geq k}]\Big| \leq \epsilon.
	\]
	Notice that for any $j$, ``$A_N$ is true and $A_{N,j}$ is false"
	if and only if $\mathrm{Pr}\big(Y_{N,\geq j} \geq 1\big)$. 
	Also, by Markov inequality, 
	$\mathrm{Pr}\big(Y_{N,\geq j} \geq 1\big)
	\leq \mathrm{E}\big[Y_{N,\geq j}\big]$.
	In consequence, for any $j\geq k$
	\[
	\LN \Big|\mathrm{Pr}\big(A_N\big)- \mathrm{Pr}\big(A_{N,j}\big)\Big|
	\leq \epsilon.
	\]
	In consequence, for any $j\geq k$
	\[ \Big| \LN \mathrm{Pr}\big(A_N \big) - F_{d,j} \Big|
	\leq  \Big| \LN \mathrm{Pr}\big(A_{N,j} \big) - F_{d,j} \Big| +
	\epsilon = \epsilon,	
	\]
	and we are finished. 
\end{proof}
	\noindent\rule{2cm}{0.4pt}\par

\begin{theorem} 
	Let $0<c<(d-2)!$.
	Let $H$ be an hypergraph whose components are all 
	unicycles, and let  $F_d:=F_d(c)$ be as in last theorem. 
	Then
	\[
	\LN \mathrm{Pr}\Big(F_N\simeq H\Big)=F_d(c)
	s^{e(H)} \frac{(d-2)!^{e(H)}}{|Aut(H)|}
	\],
	where 
	\[
	s=\frac{c}{(d-2)!}e^{\frac{c}{(d-2)!}},
	\]
	as before.
\end{theorem}

\begin{proof}
	Fix such $H$.
	Let $U_1,U_2,\dots, U_i,\dots$ be an enumeration of all unicycles.
	For each $i\in \N$, let $a_i$
	be the number of connected components of $H$ that are isomorphic to $U_i$,
	and let $W_{N,i}$ be the random variable that counts the number of connected 
	components in $G^d(N,P(N))$ that are isomorphic to $U_i$. Clearly,
	$F_N\simeq H$ if and only if $W_{N,i}=a_i$ for all $i$.  Thus,
	\[
	\LN \mathrm{Pr}\big(F_N\simeq H\big)= 
	\LN \mathrm{Pr}\big( \bigwedge_{i=1}^\infty W_{N,i}=a_i\big).
	\]
	First, we are going to show that 
	\[
	\LN \mathrm{Pr}\big( \bigwedge_{i=1}^\infty W_{N,i}=a_i\big)=
	\lim\limits_{j\to \infty}
	\LN \mathrm{Pr}\big( \bigwedge_{i=1}^j W_{N,i}=a_i\big).
	\]
	Fix $\epsilon > 0$ an arbitrarily small real constant. We need
	to prove that there exists some $j_0\in \N$ satisfying
	\[
	\left|\LN \mathrm{Pr}\big( \bigwedge_{i=1}^j W_{N,i}=a_i\big)
	- \lim\limits_{j\to \infty}
	\LN \mathrm{Pr}\big( \bigwedge_{i=1}^j W_{N,i}=a_i\big)
	\right|
	\leq \epsilon
	\text{ for all } j\geq j_0.
	\]
	For each $k \in \N$ let $X_{N,k}$ be the random variable
	that counts the uni-cyclic connected components of $G^d(N,p(N))$
	with exactly $k$ edges.	By \cref{thm:sizefragment}
	we have that for some $k_0\in \N$
	\[ 
	\LN \sum_{l=k_0}^{\infty }\mathrm{E}\big[ X_{N,l} \big] \leq 
	\epsilon
	\]
	Let $k_1$ be the maximum number of edges in a connected component of 
	$H$, and let $k = \max(k_0, k_1+1)$. Finally, fix $j_0$ such that 
	$e(U_j)>k_1$ for any $j\geq j_0$. \par
	Given any $j\geq j_0$, $F_N \simeq H$ if and only if 
	\[
	\big(\bigwedge_{i=1}^{j} W_{N,i}=a_i\big) \wedge
	\big(\sum_{l=k}^\infty X_{N,l}=0  \big) 
	\]
	Using Markov inequality we get 
	\[
	\LN \mathrm{Pr}\big( \sum_{l=k}^\infty X_{N,l}\geq 1  \big) \leq \epsilon
	\]
	Thus,
	\[
	\left|\LN \mathrm{Pr}\big( \bigwedge_{i=1}^j W_{N,i}=a_i\big)
	- \lim\limits_{j\to \infty}
	\LN \mathrm{Pr}\big( \bigwedge_{i=1}^j W_{N,i}=a_i\big)
	\right|
	\leq \epsilon,
	\]
	as we wanted to prove. 	\par
	Using \cref{thm:probunicycliccomponents} we get
	\[
	\lim\limits_{j\to \infty}
	\LN \mathrm{Pr}\big( \bigwedge_{i=1}^j W_{N,i}=a_i\big)=
	\prod_{i=1}^\infty e^{-\lambda_i} \frac{\lambda_i^{a_i}}{a_i!},
	\]
	where $\lambda_i=s^{e(U_i)}\frac{(d-2)!^{e(U_i)}}{|Aut(U_i)|}= 
	\LN \mathrm{E}[W_{N,i}]$, and $s=s(c)$ is defined as in the statement 
	of the theorem. Using \cref{thm:limitchangeunicycles} we obtain
	\[
	\sum_{i=1}^{\infty} \lambda_i = B_d(c),
	\]
	and in consequence 
	\[
	\prod_{i=1}^\infty e^{-\lambda_i}= e^{-B_d(c)}=A_d(c).
	\]
	The following identities hold:
	\[
	\sum_{i=1}^\infty e(U_i)a_i = e(H) \qquad 
	\prod_{i=1}^{\infty} |Aut(U_i)|^{a_i} a_i!= |Aut(H)|.
	\]
	As a consequence we get
	\[
	\prod_{i=1}^\infty \frac{\lambda_i^{a_i}}{a_i!}=
	\prod_{i=1}^\infty (s(d-2)!)^{e(U_i)a_i} 
	\frac{1}{|Aut(U_i)|^{a_i}a_i!}=
	s^{e(H)}\frac{(d-2)!^{e(H)}}{|Aut(H)|},
	\]
	and the theorem follows. 
\end{proof}
	\noindent\rule{2cm}{0.4pt}\par
Let us denote by $\mathcal{U}$ the class of hypergraphs
whose connected components are unicycles. For the case $d=2$ 
the asymptotic distribution of the fragment $F_N$ coincides with
the Boltzmann-Poisson distribution of random graphs from $\mathcal{U}$
described in \cite{mcdiarmid2009random}.
\todo[inline]{Tobias de una forma un poco críptica da a entender que
	usar el teorema 1.3 de \cite{mcdiarmid2009random} sirve para algo. Yo no he sabido cómo
	o para qué.}
\par
For completeness sake we give an argument of why the asymptotic distribution
of $F_N$ is indeed a probability distribution.  



\begin{theorem} \label{thm:limitdistribution}
	Let $0<c<(d-2)!$. 
	Let $H_1, \dots H_k, \dots$ be an enumeration of all hypergraphs
	in $\mathcal{U}$. For each $i\in \N$ 
	denote by $p_i$ the limit $\LN \mathrm{Pr}(F_N \simeq H_i)$. Then
	\[
	\sum_{i=1}^\infty p_i = 1.	
	\]
\end{theorem}
\begin{proof}
	Let $\epsilon>0$ be an arbitrarily small real constant. We show that 
	there exists some $j_0\in \N$ such that
	\[
	1- \sum_{i=1}^j p_i \leq \epsilon \quad \text{ for all } j\geq j_0.	
	\]	
	Let $m=\LN \mathrm{E}[e(F_N)]$. Notice that $m$ exists by \cref{thm:sizefragment}.
	Define $M=m/\epsilon$. Then 
	\[
	\LN \mathrm{Pr}\big(e(F_N)\geq M\big)\leq \epsilon.
	\]
	Let $j_0$ be such that $e(H_i)\geq M$ for all $i\geq j_0$. 
	Then, given any $j\geq j_0$, 
	\[
	1 - \sum_{i=1}^j p_i = \LN \mathrm{Pr}\big(
	\bigwedge_{i=1}^j F_N \not\simeq H_i\big) \leq
	\LN \mathrm{Pr}\big(
	e(F_N) \geq M \big)\leq \epsilon, 	
	\]	
	and the result follows. 
\end{proof}

Given $H\in \mathcal{U}$, the property
$F(G)\simeq H$ cannot be expressed in FO logic. This is
because one has to rule out the existence of arbitrarily 
long cycles in $G$. Hence, in the following proof 
it will be useful to consider "the fragment up
to connected components with $i$ edges".
Given an hypergraph $G$ and $i\in \N$, let $F_i(G)$
be the union of connected components
in $G$ that are unicycles with no more than $i$ edges, and let
$\mathcal{U}_i$ be the class of hypergraphs whose connected components
are unicycles with at most $i$ edges. 
Given any $H\in \mathcal{U}_i$, the property $F_i(G)\simeq H$, 
unlike $F(G)\simeq H$, is expressible in FO logic. We will 
abbreviate $F_i(G_N)$ as $F_{N,i}$. 

\begin{theorem}
	Let $0<c<(d-2)!$. Let 
	$H_1,\dots, H_i,\dots$ be an enumeration of all
	hypergraphs in $\mathcal{U}$ and for each $i\in \N$ 
	let $p_i=\LN \mathrm{Pr}(F_N\simeq H_i)$. 
	Consider the sets 
	\[
	L_c:=\{
	\LN \mathrm{Pr}\big(P(G_N)\big) \, | \,
	P \text{ FO property} \},
	\] 
	and 
	\[
	S_c:=\{
	\sum_{i\in T} p_i \, | \, T \subseteq \N \,
	\}.	
	\]
	Then it is satisfied that $\overline{L_c}=S_c$. 		
\end{theorem}
\begin{proof}
	We will prove the statement by showing both 
	$S_c\subseteq \overline{L_c}$ and
	$\overline{L_c}\subseteq S_c$. \par 
	We begin with 
	$S_c\subseteq \overline{L_c}$. 
	Let $T\subset \mathcal{U}$, and let $\epsilon>0$ be
	an arbitrarily small real number. We show that 
	there exists a first order property $P$ such that
	\[ \Big| \LN \mathrm{Pr}\big( P(G_N)\big) 
	- \sum_{i\in T} p_i \Big| \leq \epsilon. \]
	As $\sum_{i=1}^{\infty} p_i=1$, there exists some $j$ such that
	$\sum_{i=1}^j p_i \leq \epsilon/2$. Fix such $j$. Let
	$T^\prime= T \cap \{ 1,\dots,j-1\}$. Suppose
	$T^\prime=\{i_1,\dots,i_k\}$. 
	Consider the properties $Q$ and $Q^\prime$ defined as
	\[
	Q(G):= \bigvee_{i\in T} F(G)\simeq H_i,  \qquad
	Q^\prime(G):= \bigvee_{x=1}^k F(G)\simeq H_{i_x}.
	\]
	Then, it is satisfied  
	\[
	\LN \Big| \mathrm{Pr}\big( Q(G_N) \big)
	- \mathrm{Pr}\big( Q^\prime(G_N) 
	\big)\Big| \leq \sum_{i=j}^\infty p_i \leq \frac{\epsilon}{2}.
	\]
	Let $l_1$ be 
	the maximum number of edges in a connected component belonging
	to any of $H_{i_1},\dots,H_{i_k}$. Let 
	$M=\LN \mathrm{E}[e(F_N)]$, and let $l_2=M/2k\epsilon$.
	Define $l=\max(l_1,l_2)$. 
	The first order property $P$ will be defined as 
	\[
	P(G):=\bigvee_{x=1}^k F_l(G)\simeq H_{i_x}.
	\]
	As $l\geq l_1$, all
	$H_{i_1},\dots, H_{i_k}$ belong to $\mathcal{U}_l$.
	For each $i_x$, we have that
	\[ \mathrm{Pr}\big(F_{N,l}\simeq H_{i_x} \wedge 
	F_N\simeq H_{i_x} \big)\leq
	\mathrm{Pr}\big(F_N\notin \mathcal{U}_l\big) \leq 
	\mathrm{Pr}\big(e(F_N)>l\big).
	\]
	and because $l\geq l_2$,
	\[
	\LN \mathrm{Pr}\big(e(F_N)>l\big)
	\leq \frac{\epsilon}{2k}.
	\]
	Thus,
	\[
	\big|\LN \mathrm{Pr}\big(P(G_N)\big)- \sum_{i\in T}p_i \big|
	= \LN \big|\mathrm{Pr}\big(P(G_N)\big) -
	\mathrm{Pr}\big(Q(G_N)\big)\big|	\leq
	\LN	\big| \mathrm{Pr}\big(P(G_N)\big) -
	\mathrm{Pr}\big(Q^\prime(G_N)\big)\big|
	+ \frac{\epsilon}{2}.
	\]
	And because $P\subset Q^\prime$, 
	\[
	\mathrm{Pr}\big(P(G_N)\big) -
	\mathrm{Pr}\big(Q^\prime(G_N)\big)
	=
	\mathrm{Pr}\big(P(G_N) \wedge \neg Q^\prime(G_N) \big).
	\]
	It is also satisfied that
	\[
	P(G_N) \wedge \neg Q^\prime(G_N) \quad \iff \quad 
	\bigvee_{x=1}^k F_{N,l}\simeq H_{i_x} \wedge 
	F_N\not\simeq H_{i_x} 
	\]
	Taking into account that the events 
	$F_{N,l}\simeq H_{i_x} \wedge 
	F_N\not\simeq H_{i_x}$ are disjoint for each $x$
	we get
	\[
	\mathrm{Pr}\big(\bigvee_{x=1}^k F_{N,l}
	\simeq H_{i_x} \wedge 
	F_N\not\simeq H_{i_x} \big)=
	\sum_{x=1}^{k} \mathrm{Pr}\big( F_{N,l}
	\simeq H_{i_x} \wedge 
	F_N\not\simeq H_{i_x} \big)
	\]
	In consequence,
	\[
	\big|\LN \mathrm{Pr}\big(P(G_N)\big)- \sum_{i\in T}p_i \big|
	\leq
	\LN	\big|\sum_{x=1}^{k} \mathrm{Pr}\big( F_{N,l}
	\simeq H_{i_x} \wedge 
	F_N\not\simeq H_{i_x} \big)|
	+ \frac{\epsilon}{2}\leq
	\sum_{x=1}^{k} \frac{\epsilon}{2k} + \frac{\epsilon}{2}
	\leq \epsilon,
	\]
	and we are finished. \par
	Now it is left to prove that
	$\overline{L_c}\subseteq S_c$. 
	It is a known fact \cite{kakeya1914partial}, 
	\cite{hornich1941beliebige}, \cite{nymann2000paper}
	that $S_c$ is a perfect set. In particular
	$S_c$ is closed and $\overline{S_c}=S_c$. In
	consequence it suffices to show that
	$L_c\subset S_c$.\par
	Let $H\in \mathcal{U}$.
	One can show via EF games (see \cite{mastertesisalberto}) 
	that for any given FO property $P$,
 	\[
 	\LN \mathrm{Pr}\big( P(G_N)) \, | \, F_N\simeq H \big)
 	= 0 \text{ or } 1.
 	\]
 	We want to show that 
 	\begin{equation}\label{eqn:aux}
 	\LN \mathrm{Pr}\big( P(G_N) \big)
 	= \sum_{i=1}^\infty \LN \mathrm{Pr}\big( F_N \simeq H_i \big)
 	\mathrm{Pr}\big( P(G_N)) \, | \, F_N\simeq H_i \big).
 	\end{equation}
 	Fix an arbitrarily small real constant $\epsilon>0$. We
 	need to prove that there exists an index $j_0\in \N$ such
 	that for all $j\geq j_0$
 	\[
 	\big|\LN \mathrm{Pr}\big( P(G_N) \big) -
 	\sum_{i=1}^j \LN \mathrm{Pr}\big( F_N \simeq H_i \big)
 	\mathrm{Pr}\big( P(G_N) \, | \, F_N\simeq H_i \big)\big|\leq 
 	\epsilon.
 	\]
	Notice that the events $F_N\simeq H_i$ are disjoint for each
	$i$. So we obtain:
	\[
	\LN \mathrm{Pr}\big( P(G_N)\big)=
	\LN \sum_{i=1}^\infty \mathrm{Pr}\big( F_N \simeq H_i \big)
	\mathrm{Pr}\big( P(G_N)) \, | \, F_N\simeq H_i \big).	
	\] 
	Let $M=\LN \mathrm{E}(e(F_N))$ and let $l=M/\epsilon$.
	There exists some $j_0\in \N$ such that $e(H_i)\geq M$ 
	for all $i\geq j_0$. In consequence, for all $j\geq j_0$,
	\[
	\LN \sum_{i=j}^{\infty} \mathrm{Pr}\big( F_N \simeq H_i \big) \leq 
	\epsilon.	
	\]
	And we obtain
	\[
	\big|\LN \mathrm{Pr}\big( P(G_N) \big) -
	\sum_{i=1}^j \LN \mathrm{Pr}\big( F_N \simeq H_i \big)
	\mathrm{Pr}\big( P(G_N) \, | \, F_N\simeq H_i \big)\big|=
	\LN \sum_{i=j}^{\infty} \mathrm{Pr}\big( F_N \simeq H_i \big) \leq 
	\epsilon,
	\]
	as we wanted. This proves \cref{eqn:aux}. Finally, let 
	\[
	T=\{i\in \N \, | \, \LN \mathrm{Pr}
	\big( P(G_N) \, | \, F_N\simeq H_i \big) = 1 \,	
	\}
	\]
	It is satisfied that 
	\[
	\sum_{i=1}^\infty \LN \mathrm{Pr}\big( F_N \simeq H_i \big)
	\mathrm{Pr}\big( P(G_N)) \, | \, F_N\simeq H_i \big)=
	\sum_{i\in T} \LN \mathrm{Pr}\big( F_N \simeq H_i \big)
	= \sum_{i\in T} p_i.
	\]
	This finishes the proof of the theorem.  
\end{proof}



\bibliography{biblio}
\bibliographystyle{unsrt}
\end{document}